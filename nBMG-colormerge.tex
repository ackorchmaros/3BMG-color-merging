\documentclass[final,3p,times]{elsarticle}
\usepackage{amsmath, amssymb, amsfonts, amsfonts, amsthm,latexsym}
\DeclareSymbolFont{yhlargesymbols}{OMX}{yhex}{m}{n}
\DeclareMathAccent{\overarc}{\mathord}{yhlargesymbols}{"F3}
\usepackage[mathscr]{euscript}
\usepackage[noend]{algorithmic}
\usepackage{comment}
\usepackage{algorithm}
\renewcommand{\algorithmiccomment}[1]{\hfill$\rhd$\textit{#1}}
\usepackage{graphics}
\usepackage{enumerate}
\usepackage{enumitem}
\usepackage[usenames]{color}
\usepackage{mathtools}
\usepackage[normalem]{ulem}
\usepackage{import}
\usepackage{comment}
\usepackage{wrapfig}
\usepackage{optidef}
\usepackage{epsfig}
\usepackage{tikz-cd}
\usepackage{color}
\DeclareMathOperator{\Tr}{Tr}
% Plots
\usepackage{graphicx}
\usepackage{float}
\usepackage[font=scriptsize,labelfont=bf]{caption}
\usepackage{subcaption}
\usepackage{url}
\usepackage{tikz}
\usetikzlibrary{positioning,chains,fit,shapes,calc}
\usepackage[all]{xy}

\newtheorem{theorem}{Theorem}[section]
\newtheorem{proposition}[theorem]{Proposition}%[section]
\newtheorem{lemma}[theorem]{Lemma}%[section]
\newtheorem{question}{Question}
\newtheorem{definition}{Definition}[section]
\newtheorem{fact}{Observation}[section]
\newtheorem{example}{Example}[section]
\newtheorem{corollary}[theorem]{Corollary}%[section]
\newtheorem{corollary*}{Corollary}%[section]
\newtheorem{remark}[theorem]{Remark}%[section]
\newtheorem{problem}{Problem}[section]
\newtheorem{observ}{Observation}

\newcommand{\TODO}[1]{\begingroup\color{red}#1\endgroup}
\def\TODOO#1{\marginpar{\tiny\raggedright\color{red}#1}}

\newcommand{\ak}[1]{\begingroup\color{orange}#1\endgroup}
\newcommand{\OLD}[1]{\begingroup\tiny\color{gray}#1\endgroup}
\newcommand{\PFS}[1]{\begingroup\color{magenta}#1\endgroup}
\newcommand{\mh}[1]{\begingroup\color{blue}#1\endgroup}

%\journal{Discrete Applied Mathematics}
%\journal{Journal of Combinatorial Theory, Series B}

\DeclareMathOperator{\lca}{lca}

\begin{document}

\begin{frontmatter}
  \title{How merging colors characterizes 3-BMGs}

  \author[LEI]{Annachiara Korchmaros}
  \ead{annachiara@bioinf.uni-leipzig.de}

  \author[STOCK]{Marc Hellmuth}
  \ead{marc.hellmuth@math.su.se}

  \author[LEI,LEI-other,MIS,TBI,BOG,SFI]{Peter F. Stadler}
  \ead{stadler@bioinf.uni-leipzig.de}

\address[LEI]{Bioinformatics Group, Department of Computer Science \&
  Interdisciplinary Center for Bioinformatics, Universit{\"a}t Leipzig,
  H{\"a}rtelstra{\ss}e 16-18, D-04107 Leipzig, Germany}

\address[STOCK]{Department of Mathematics, Faculty of Science,
Stockholm University, SE-10691 Stockholm, Sweden}

\address[LEI-other]{German Centre for Integrative Biodiversity Research
  (iDiv) Halle-Jena-Leipzig, Competence Center for Scalable Data Services
  and Solutions Dresden-Leipzig, Leipzig Research Center for Civilization
  Diseases, and Centre for Biotechnology and Biomedicine at Leipzig
  University at Universit{\"a}t Leipzig}

\address[MIS]{Max Planck Institute for Mathematics in the Sciences,
  Inselstra{\ss}e 22, D-04103 Leipzig, Germany}

\address[TBI]{Institute for Theoretical Chemistry, University of Vienna,
  W{\"a}hringerstrasse 17, A-1090 Wien, Austria}

\address[BOG]{Facultad de Ciencias, Universidad National de Colombia, Sede
  Bogot{\'a}, Colombia}

\address[SFI]{Santa Fe Institute, 1399 Hyde Park Rd., Santa Fe NM 87501,
  USA}


\begin{abstract}
  
\end{abstract}


\begin{keyword}
 Colored directed graphs; rooted trees; phylogenetic
  combinatorics; best matches
\end{keyword}

\end{frontmatter}

\sloppy

\section{Introduction}

\section{Background}
\label{sec:background}
\subsection{Vertex-colored digraphs}


We consider \emph{directed graphs} $G=(V,E)$ with vertex set $V(G)\coloneqq V$
and edge set $E(G)\coloneqq E$ where $E\subseteq (V\times V)\setminus
\{(x,x)\mid x\in V\}$. Hence, by definition, $G$ does not contain loops or
multiple edges. We denote edges $e=(x,y)\in E$ simply by $xy$. 
An edge $xy\in E$ is \emph{symmetric} if $yx\in E$. 

For a vertex $x\in V$, the out-neighbors of $x$ in $G$
are those vertices $z\in V$ for which $xz\in E$, respectively. The 
out-neighborhood of $x$, i.e., the set of out-neighbors of $x$ is denoted by $N_G(x)$, or 
simply by $N(x)$ if there is no risk of confusion. 
A digraph $G$ is called sink-free if each of its vertices has at least one out-neighbor.

Let $S$ be a finite set of colors. A digraph $G=(V,E)$ is colored by the colors
in $S$, if there is a surjective map $\sigma: V\to S$ where $\sigma(x)$ is
called the \emph{color} of $x$. We use $(G,\sigma)$ to specify that $G$ is
equipped with such a map $\sigma$ and often say that $G$ is an $|S|$-colored
graphs to specify the number of colors used in $G$. A colored digraph $(G,
\sigma)$ is \emph{properly colored} or, equivalently, $\sigma$ is a \emph{proper
coloring} of $G$, if $\sigma(x)\neq \sigma(y)$ for all $xy\in E$. 

For every color $s\in S\setminus \{\sigma(x)\}$ we write $N(x,s)$ for the sets
of out-neighbors of $x$ with color $s$, respectively. We say that $(G,\sigma)$
is \emph{color-sink-free} if $N(x,s)\ne\emptyset$ for all $x\in V(G)$ and all
$s\in \sigma(V)\setminus\{\sigma(x)\}$.


\subsection{Rooted trees and triples, and leaf-colored trees} 
We consider rooted trees $T$. The root of $T$ is a distinguished vertex
$\rho_T\in V(T)$. For two vertices $x,y\in V(T)$, we write $y \preceq_{T} x$ if
$x$ lies on the unique path from the root to $y$, in which case $x$ is called an
\emph{ancestor} of $y$, and $y$ is called a \emph{descendant} of $x$. Note that
edges $e=xy\in E(T)$ imply that $y\preceq_T x$. In the latter case, $y$ is a
\emph{child} of $x$. Moreover, we say that $x$ and $y$ are \emph{comparable} if
$y\preceq_{T} x$ or $x\preceq_{T} y$ holds and, otherwise, $x$ and $y$ are
\emph{incomparable}. Note that $\preceq_{T}$ is a partial order with a unique
maximal element $\rho$. The \emph{leaves} $L=L(T)\subseteq V(T)$ of $T$ are
precisely the $\preceq_{T}$-minimal elements.

From here on, we assume that the root $\rho_T$ as well as every non-leaf vertex
of a tree have always at least two children. 

For a set of leaves $A\subseteq L$, we write $\lca_T(A)$ for the the \emph{last
common ancestor} of $A$, i.e., the unique $\preceq_T$-minimal vertex in $V(T)$
such that $x\preceq \lca_T(A)$ for all $x\in A$. For simplicity, we write
$\lca_T(x,y)$ instead of $\lca_T(\{x,y\})$. 

A (rooted) triple is a binary rooted tree on three vertices.  We denote by
$xy|z$ the rooted triple $t$ with leaf set $\{x,y,z\}$ and
$\lca_t(x,y) \prec_T \lca_t (x,z) = \lca_t(y,z)$. A tree $T$ \emph{displays}
$xy| z$ if $\lca_T(x,y) \prec_T \lca_T (x,z) = \lca_T(y,z)$.  


Two rooted trees $T_1=(V_1,E_1)$ and $T_2=(V_2,E_2)$ are \emph{isomorphic}, if
there is a bijection $\varphi:V(T_1) \rightarrow V(T_2)$ such that
$\varphi(\rho_{T_1}) = \rho_{T_2}$ and, in addition, $xy\in E_1$ if and only if
$\varphi(x)\varphi(y)\in E_2$. 


We consider trees $(T,\sigma)$ that are equipped with a coloring if its leaves, 
i.e., a surjective map $\sigma: L(T)\to S$.  


\subsection{Best Match Graphs}

Given a leaf-colored rooted tree $(T,\sigma)$, a leaf $y\in L(T)$ is a
{\emph{best match of}} $x\in L(T)$, in symbols $x\rightarrow y$, if
${\lca_T}(x,y)\preceq {\lca_T}(x,y')$ for all $y'\in L$ with
$\sigma(y')=\sigma(y)$. The \emph{best match graph} (\emph{BMG})
$\mathcal{G}(T,\sigma)$ associated to $(T,\sigma)$ is the digraph with vertex
set $L$ where the edges are the ordered pairs $xy$ with $x\rightarrow y$ and
$x\neq y$. A colored digraph $(G, \sigma)$ is a \emph{BMG}, whenever there is a
tree $(T,\sigma)$ such that $(G, \sigma) = \mathcal{G}(T,\sigma)$. In this case,
$(T,\sigma)$ \emph{explains} $(G, \sigma)$. 
    


\begin{definition}
Let $(G,\sigma)$ be a vertex-colored digraph. Then the set of
  \emph{informative triples} is
  \begin{align*}
    \mathscr{R}(G,\sigma) \coloneqq \big\{&    ab|b' \colon
    \sigma(a)\neq\sigma(b)=\sigma(b'),\,    ab\in E(G), \text{ and }
    ab'\notin E(G) \big\},
    \intertext{and the set of \emph{forbidden triples} is}
    \mathscr{F}(G,\sigma) \coloneqq \big\{
    &ab|b' \colon
    \sigma(a)\neq\sigma(b)=\sigma(b'),\,
    b\ne b',\, \text{ and }
    ab,ab'\in E(G) \big\}.
  \end{align*}
\end{definition}

\begin{proposition}[\cite{Schaller:21b}, Prop.~3.4 and Thm.~3.5]
  \label{prop:BMG-triple-charac}
  A properly colored graph $(G,\sigma)$ is a BMG if and only if (i) it is
  color-sink-free and (ii) $(\mathscr{R}(G,\sigma),\mathscr{F}(G,\sigma))$
  is consistent. 
  
  In particular, a leaf-colored tree $(T , \sigma)$  explains $(G, \sigma )$ 
  if and only if $(T , \sigma)$ agrees with $(\mathscr{R}(G,\sigma),\mathscr{F}(G,\sigma))$. 
\end{proposition}

\section{Color Merging}
\label{sec:operators}


%\OLD{
%\begin{definition} \TODO{below is a unified version of these two types defined here! **** }
%\label{def:color-derivative}
%	Let $(G,\sigma)$ be an $n$-colored digraph and $S'\subseteq S$. We
%    define $\sigma_{|S'}\colon V(G) \to S'\cup \{c\}$ where $c\notin S$ by
%    \[\sigma_{|S'}(x)=c \iff \sigma(x)\notin S', \text{ for all } x\in V(G).\] 
%    Hence, $\sigma_{|S'}$ keeps the colors of all vertices with
%    color $i\in S'$ while all other vertices obtain a common color $c$ that is
%    not contained in $S$. 
%    Moreover, we define a 2-coloring 
%    $\sigma^2_{|S'}\colon V(G) \to \{c',c\}$ where $c,c'\notin S$ by
%    \[\sigma_{|S'}(x)=c \iff \sigma(x)\notin S', \text{ for all } x\in V(G).\] 
%    In this case, all vertices in $S'$ obtain color $c'$ while all vertices with a color in $S\setminus S'$ 
%    obtain color $c$.\smallskip
%    
%    \noindent
%    We define the digraph $G_{|S'}$ by putting
%\begin{enumerate}
%    \item $V(G_{|S'}) \coloneqq V(G)$ and
%    
%    \item $xy\in E(G_{|S'})$  presicely if $xy\in E(G)$ and 
%    	  $\sigma(x)\in S'$ or  $\sigma(y)\in S'$.
%    	  
%    	  In other words, the edges of $G_{|S'}$ are precisely those edges 
%    	  of $G$ that are incident to a vertex with color $i\in S'$.
%\end{enumerate}

%	\noindent 
%    We define the digraph $G^2_{|S'}$ by putting
%	\begin{enumerate}
%    \item $V(G^2_{|S'}) \coloneqq V(G)$ and
%    
%    \item $xy\in E(G^2_{|S'})$ presicely if $xy\in E(G)$ and 
%    	  $\sigma^2_{S'}(x) \neq \sigma^2_{S'}(y)$.
%    	  
%    	  In other words, $G^2_{|S'}$ 
%    	  is obtained from $(G,\sigma)$ by removing all
%    	  edges between vertices $x$ and $y$ that have 
%    	  either both a color in $S'$ or in $S\setminus S'$
%\end{enumerate}

%%
%	The \emph{$S'$-color derivative} $\partial_{|S'}(G,\sigma)$ is the
%	$(|S'|+1)$-colored digraph $(G_{|S'},\sigma_{|S'})$. 
%	The \emph{$(S',2)$-color derivative} $\partial^2_{|S'}(G,\sigma)$ is the
%	$2$-colored digraph $(G^2_{|S'},\sigma^2_{|S'})$. \smallskip
%	
%	\noindent
%	In case that $S' = \{i\}$, we replace, for simplicity, ``$S'$'' by ``$i$'',
%	in which case, the $i$-color derivative $\partial_{|i}(G,\sigma)$ is
%	the the 2-colored digraph $(G_{|i},\sigma_{|i})$.
%	"Analog 	 $(G^2_{|i},\sigma^2_{|i})$ instead."

%%\TODO{\small MH: if we see that we dont need this general def, we may replace it at the end 
%%just be $i$-color derivative $\partial_{|i}(G,\sigma)$ etc }
%\end{definition} 

%maybe short arguments, that for $S' = \{i\}$, the graphs $(G^2_{|i},\sigma^2_{|S'})$
%and $(G_{|i},\sigma_{|i})$ are identical up to recoloring vertices with color $c'$
%in $(G^2_{|S'},\sigma^2_{|S'})$ by color $i$.
%}	
	
	
\begin{definition}
	Let $(G,\sigma)$ be an $n$-colored digraph and $\Pi = \{S_1,\dots,S_k\}$, $k\geq 1$
	be a partition of $S$. The $k$-colored graph $(G_{\Pi},\sigma_{\Pi})$ is defined as follows. 
	
	
	Define $\sigma_{\Pi}\colon V(G) \to \{c_1,\dots,c_k\}$ by putting, for all $x\in V(G)$, 
    \[\sigma_{\Pi}(x)=c_i \iff x\in S_i\in \Pi..\] 
    
    Define  $G_{\Pi}$ by putting
	\begin{enumerate}
    	\item $V(G_{\Pi}) \coloneqq V(G)$ and
    
    \item $xy\in E(G_{\Pi})$  presicely if $xy\in E(G)$ and 
    	  $\sigma_{\Pi}(x)\neq \sigma_{\Pi}(y)$. 
    	  
    	  In other words,  $G_{\Pi}$ 
    	  is obtained from $(G,\sigma)$ by removing all
    	  edges between vertices $x$ and $y$ that are 
    	  in a common class $S_i\in \Pi$. 
\end{enumerate}
\end{definition}	
	
	



\TODO{MH:what are necassary/sufficient conditions on $\Pi$ and $(G_{\Pi},\sigma_{\Pi})$
such that we get $(\mathscr{R}(G,\sigma),\mathscr{F}(G,\sigma))$ for color-sink free $(G,\sigma)$
? as this lies at the heart of all the 3BMG proofs as well. }

















\bibliographystyle{elsarticle-num}
\bibliography{references}

\end{document}















